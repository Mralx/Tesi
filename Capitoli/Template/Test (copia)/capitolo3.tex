\chapter{Impostazione del problema di ricerca}
\label{capitolo3}
\thispagestyle{empty}

\section{Exploration problem}

The exploration problem can be defined as the problem of mapping an
unknown environment by means of a team of robots. Thus, the objective
is the construction of a map of the environment but this introduces
further problems, concerning for example the localization of each
robot dealing with the various sources of uncertainty, how to distribute
the team, and how to coordinate them. These are three of the main
aspects considered in the development of a solution to this problem,
providing the major impact on performance. To compare different approaches
the main metric used is the time taken by the exploration \cite{Burgard2005,Rogers2013,Cattaneo2017},
but also the distance traveled by the robots is sometimes considered
\cite{Amigoni2008}.

A formal definition of the \emph{exploration problem}, in the following
abbreviated as \emph{EP},\emph{ }can be provided through a 4-ple $\left\langle A,E,P_{0},T\right\rangle $
where: 
\begin{itemize}
\item $A$ is the set of robots used for the exploration;
\item $E$ is the environment to explore;
\item $P_{0}$ is a vector of the initial poses of the robots of the team;
\item $T$ is the termination criterion.
\end{itemize}
This four elements characterize the instance of EP addressed, while
the solution of the same consists in producing a map $M$ of the environment
$E$. To do this, each robot is able to perceive the environment through
one or more sensors, characterized by the size of the spanned area.
The map $M$ consists in a 2D occupancy grid representing the areas
of $E$. The center of each cell $c$ is characterized by its coordinates
in a global coordinate system, provided as a vector $(x_{c},y_{c})$.
Each cell contains a value which tells if that cell is free, occupied
by an obstacle, or still unknown. Clearly, the first two values refer
only to already explored cells, while the third value refers to those
cells which are still not scanned by any robot of the team. 

The experiments presented in this work are run on MRESim, a 2D simulator
aimed at testing multi-robot exploration scenarios \cite{Spirin2015}.
In particular, it allows to create an instance of \emph{EP }through
a configuration file which provides the robots in the team, their
initial poses and a PNG image representing the environment to explore.
The vectors $P_{0}$ employed have in common the disposition of the
robots along a straight line suitably spaced to make them start from
the same initial location. The termination criterion used is the percentage
of explored area and it can be modified by varying a variable in the
code.

The following sections describe the system used for the experiments,
focusing at first on the environment and the agents, then moving the
attention to the exploration strategy and the coordination mechanisms
considered in this thesis. 

\section{Localization and mapping}

The problem of mapping an unknown environment and simultaneously localize
a single robot on the partial map built is one of the fundamental
problems of robotics, called \emph{SLAM problem,} from \emph{Simultaneous
Localization And Mapping. }The robot knows its initial pose and, as
it moves, the uncertainty about its pose increases, due to uncertainty
in the odometry. For this reason it becomes necessary to localize
the robot on the map, even a partial one. A formal description of
it, as provided in \cite{Thrun2008}, is suitably done
in a probabilistic framework and distinguishes two versions, the \emph{online
}and the \emph{full} \emph{SLAM problem}. Let $X_{T},U_{T}$, and
$Z_{T}$ be three statistical variables representing respectively
the sequence of poses assumed by the robot, i.e., the \emph{path},
the sequence of odometry measurements, and the sequence of measurements
provided by the sensors. Let also $m$ be the true map of the environment.
The \emph{full SLAM problem }is then defined as the problem of estimating
the posterior probability of the map together with the whole path
traveled by the robot, that is $p\left(X_{T},m|Z_{T},U_{T}\right)$.
The \emph{online SLAM problem }aims at estimating the posterior of
the actual location of the robot, rather than the whole path, together
with the map. Thus, $p\left(x_{T},m|Z_{T},U_{T}\right)$.

What happens in practice is that as the robot moves in an unknown
environment, it perceives the surroundings through its sensors, has
an estimate of its pose from its odometry measures and these are used
to reduce the uncertainty both about the map built and the location
of, or the path followed by, the robot, depending on the version of
the SLAM algorithm implemented. As the exploration goes on, the partial
map approaches to a complete map of the environment and it is also
possible to get rid of the noise in the measurements. 

The model used for the map can be both topological \cite{Grisetti2010}
or metric \cite{Montemerlo2002}. In particular in this thesis, the representation
provided as output by the SLAM algorithm is metric, consisting of
an occupancy grid. 

\subsection{Environments}

This work considers only indoor environments. Reasons behind this
restriction are related to the work of \cite{Cattaneo2017}, of which
this thesis is an extension, in order to provide a comparison in the
same type of environments. Moreover, this focus is justified by the
application contexts, being the exploration of indoor environments
more common with respect to outdoor ones.

The representation of the map provided by the SLAM algorithm used
in this work is a two-dimensional occupancy grid. Moreover, being
applied to a multi-robot scenario, it comes out the difficulty of
merging the maps computed by solving the SLAM problem for each agent.
The solution to this is a centralized approach by means of a base
station, which collects all the individual local maps and combines
them into a single global map.

Similarly to \cite{Cattaneo2017}, the environments are classified according
to some features characterizing them. In particular, the aspects considered
are the size, the openness, and the parallelizability. The first two
are easy to formally define. For the third one, a formal definition
can hardly be provided being influenced by many factors. To make the
definitions more readable, recalling that $M$ denotes the map of
the environment modeled as an occupancy grid, let $S:M\rightarrow\mathbb{R}$
be an auxiliary function which takes as input a cell of the map and
provides its area. The features considered can be defined in the following
way.
\begin{itemize}
\item Size: the amount of free area in the map. Thus, $D(M)=\sum_{i}S(f_{i})$
with $f_{i}$ being a free cell. The size of the environment clearly
affects the average time taken by robots to explore it. The smaller
the environment, the lower the amount of resources needed to map it.
From this, the distinction into small (S) and large (L) environments.
\item Openness: the property of an environment to be composed of large open
spaces (O) rather than cluttered ones (C). It is related to obstacle
density, so to the probability of a randomly picked cell to be occupied
by an obstacle. This is formally defined in a probabilistic framework
as $O\left(M\right)=\frac{\sum_{i}S(o_{i})}{\sum_{j}S(c_{j})}$where
$o_{i}$ is an occupied cell and $c_{j}$ is a generic cell.
\item Parallelizability: the property of an environment to enforce the spreading
of the robots during the exploration. A formal definition to this
is hard to provide, but it can be informally presented considering
that if the environment allows the team to spread, it is highly parallelizable
(HP). Otherwise, if the robots are forced to stick together, it is
lightly parallelizable (LP). 
\end{itemize}
To make these distinctions clearer, it is worth looking at some of
the environments used in this work and at how they are classified
according to these features. In Figure 1, two very different environments
are presented. 
\begin{figure}
\subfloat[S-O-LP environment]{\includegraphics[width=6cm]{\string"/home/alex/Scrivania/desktop/Tesi/Documenti/Marco Catteneo/MRESim/environments/Tesi/env_3\string".eps}

}\quad{}\quad{}\subfloat[L-C-HP environment]{\includegraphics[width=6cm]{\string"/home/alex/Scrivania/desktop/Tesi/Documenti/Marco Catteneo/MRESim/environments/Tesi/env_5\string".eps}

}\caption{Example of two different environments}

\end{figure}

The environment in Figure 1.a is composed of a series of large corridors,
ending in some rooms. This is classified as S-O-LP, because its dimension
is small and the areas of which it is composed are mainly open ones,
not cluttered rooms. The lightly parallelizability comes out from
its configuration, where the robots follow almost similar paths, forced
by the long corridors and they are not allowed to spread into it. 

The second one (Figure 1.b) is likely to be the map of an office.
It is composed of a lot of rooms, corridors and spaces with different
sizes. It is classified as L-C-HP because the amount of free area
is really large and is mainly composed of rooms and limited spaces.
For the configuration of the corridors, the robots tend to spread
in different directions, without sticking close to each other. For
this reason, the environment is classified as highly parallelizable.

\subsection{Agents}

In MRESim, an agent is characterized through:
\begin{itemize}
\item a number and an ID to uniquely identify it;
\item its pose, expressed by a vector $p=\left[x,y,\phi\right]$, where
$\left(x,y\right)$ are the coordinates of the grid cells occupied
by the robot and $\phi$ is its orientation;
\item the sensing range, set to the default value;
\item the communication range, assumed to be infinite;
\item the battery life, assumed to be infinite as well;
\item its type, it can be the base station, a relay, or an explorer.
\end{itemize}
The communication range and the battery life are assumed to be infinite
because the focus of this work is on the results provided by the use
of different coordination mechanisms. In this way, it is possible
to simplify the mechanisms not to consider scenarios in which the
robots run out of fuel or they are unable to communicate, even if
these situations are of particular interest in a more realistic context
and object of study as in \cite{Awerbuch1999} and \cite{Burgard2005}, respectively. 

As stated above, the type of the robot, not to be confused with its
role, can be: base station, relay, or explorer. The \emph{relay} type
is never used in the simulations performed for this work, being useful
in cases in which communication between a robot and the base station
is not possible using a direct link. Indeed, each team used is composed
of one base station and from two to eleven explorers. 

The \emph{base station} is the central coordinator of the team and
stores the global map computed by merging the partial maps provided
by the agents. It is configured similarly to other agents but its
pose is fixed to the initial one. It is important for the purpose
of exploration because allows agents to coordinate by means of a centralized
unit, rather than a decentralized approach, which would make the merging
of the map way more difficult. An example of a decentralized approach
is in \cite{Benedettelli2012}, where only as soon as two robots meet, they are
able to merge their maps and then proceed with the exploration based
on this updated map.

The \emph{explorer }is the main kind of agents employed, being the
one which moves in the environment to map it. Apart from the configuration
parameters presented above, it is also characterized by a finite speed.
Explorer are equipped with a laser range sensor allowing them to scan
a semicircle of radius equal to the sensing range in front of them.
The value for the sensing range is specified in the configuration
file. As an explorer perceives previously unknown cells with its laser
sensor, it communicates the measurements to the base station which
merges them with the current map and updates it. The updated map is
then sent to each agent. 

\section{Exploration strategy}

The frontier-based exploration presented in \cite{Yamauchi1998} is applied
to identify the candidate locations to explore. A frontier is the
boundary region between known and unknown space and by moving towards
frontiers, this boundary is pushed forward accordingly.

As highlighted in the previous section, the model for the map is an
occupancy grid. Each robot has its own local map, updated through
the sensor measurements, and merged at the base station to provide
a global map, which is the one used in the frontiers identification
phase.

Frontiers are identified by at first drawing the contour of the known
space. The contour, represented by the pink line in Figure 2, depicts
a clear distinction between the known space, the yellow one, and unknown
space, the gray one. Then, the contour is split into smaller segments
such that each segment is included within two obstacles. Due to the
representation based on occupancy grids, segments are composed of
a series of cells, whose centers are used as vertices of a polygon.
If the area of this polygon is higher than a certain threshold, then
the cell located at its barycenter is a frontier. In the figure, the
three identified frontiers are indicated with the little squares.
The presence of two frontiers in the room on the right is due to the
obstacle in the middle of that room, which splits the contour into
two segments. Another aspect to notice is that the lower frontier
on the right is located in the middle of the known space, not on the
border, and this is due to the shape of the polygon, being it convex. 

\begin{figure}
\includegraphics[width=8cm]{\string"/home/alex/Scrivania/desktop/Tesi/Capitoli/Schermata da 2020-03-29 12-50-36\string".eps}

\caption{Frontiers detected on the map}

\end{figure}


\section{Coordination mechanisms}

A coordination mechanism is the algorithm providing the allocation
of robots to the possible frontiers computed by the exploration strategy.
Together with the SLAM algorithm and the exploration strategy, this
completes the view on the system needed for the exploration. In fact,
the process of exploring an unknown environment starts with the robots
scanning an area through their sensors and these data, together with
the initial locations, are used as input to the SLAM algorithm, which
provides a partial map as output. This is given to the exploration
strategy that, as stated above, performs a discretization of the polygon
enclosing the known area and computes the frontiers, excluding the
ones too small. At this point, the coordination mechanism is applied
to find out an allocation of robots to frontiers. 

The coordination mechanisms are the main focus of this work. In the
following, the coordination mechanisms we consider are presented both
as the base mechanisms from \cite{Rogers2013} and their extensions provided
by \cite{Cattaneo2017}.

\subsection{Base mechanisms}

In \cite{Rogers2013}, three mechanisms are presented, namely \emph{reserve,
buddy system, }and \emph{divide and conquer.} They are all focused
on how \emph{proactive }the team members not strictly needed for the
exploration are. The team can be separated in two sub-teams, the \emph{active
}and the \emph{idle set},\emph{ }with the first one composed of the
robots whose goal is to explore one of the frontiers detected, while
the second one is made up by the remaining robots. An example is useful
to clarify this distinction and it is provided by Figure 3. 

This is a snapshot of the exploration just after the one of Figure
2. Five frontiers are detected: one located between the robots G and
H, one in the central corridor and three on the right. To each one
is assigned a robot, except for the three on the right. Being the
distance among them within a certain clustering threshold, only a
robot is turned into active and assigned to the nearest one. Thus,
a total of three robots is turned into active, depicted in purple
in the figure, and assigned to a frontier, while the remaining five
robots compose the idle set, depicted in blue. 
\begin{figure}[h]
\includegraphics[width=8cm]{\string"/home/alex/Scrivania/desktop/Tesi/Capitoli/Schermata da 2020-03-29 12-50-54\string".eps}\caption{Separation into active set and idle set. Purple triangles are the
robots of the active set. Each one is assigned to a different frontier
and the red line is the path from the robot to it. Blue triangles
are the robots composing the idle set. }
\end{figure}

The policy for the active set is the same for all the mechanisms,
to explore the closest frontier. The differences come out dealing
with the idle set, in fact the way in which robots are proactively
moved allows to distinguish among each method. Moreover, this theme
of a proactive use of the idle set is the leitmotiv linking the work
started in \cite{Rogers2013} and extended both by \cite{Cattaneo2017} and this
thesis.

Reserve is the less proactive mechanism because as the name implies,
each robot composing the idle set is left as a reserve at its initial
location. As the exploration starts and the first frontiers are detected,
the robots are split into the active set and the idle set, in a similar
way to the example provided before. Then, the robots in the active
set move towards their assigned frontiers and the ones in the idle
set remain still in their initial positions. As the exploration goes
on and new frontiers are found, their number may be higher than the
size of the active set, making some or all the robots from the idle
set needed, which are then turned into active and assigned to a frontier.
Once the idle set is empty and all the robots are active, the exploration
proceeds in an uncoordinated way. This means that each robot is assigned
to the closest frontier, without taking into account whether another
robot has been already assigned to it.

Divide and conquer is the most proactive mechanism presented in that
work because the idle set moves together with the active set. At the
beginning of the exploration, active agents are assigned to the frontiers
and the idle set is split in several subsets, one for each active
agent. For example, assume that at the beginning, only a frontier
is found. Then, an agent is marked as \emph{leader }and assigned to
explore it. The other robots follow it as it moves towards its frontier,
until at least another frontier is detected. For the sake of the example,
assume that at this point there are two frontiers. In this case, a
robot from the idle set is turned into active and marked as leader.
The idle set is split in two: one half follows the first leader, the
other half follows the other one. This goes on until the idle set
is empty, after which the exploration proceeds in an uncoordinated
way. Differently from the reserve mechanism, this approach allows
to have robots from the idle set nearer to the frontier to which they
are going to be assigned. In this sense, the idle team members are
considered to be more proactive with respect to reserve, where waiting
at the initial location makes the distance between the robot turned
into active and the assigned frontier higher. 

Buddy system is considered to be halfway between the two previous
mechanisms for what concerns proactivity of the agents. This comes
clearer by looking at how the mechanism handles the idle set. As soon
as robots are deployed on the environment, pairs are formed, composed
of a robot marked as \emph{leader} and another marked as \emph{follower},
each one is the \emph{buddy} of the other. Once frontiers are identified,
the minimum number of leader agents is turned into active and assigned
to them. Each follower follows its own leader towards the assigned
frontier. The other pairs remain still at the initial location, similarly
to the idle set in the reserve mechanism. When a branching point is
met, that is a point where there are two or more frontiers, the pair
is split and the leader is assigned to a frontier, while the follower
to the other one. Here, the analogy with the divide and conquer mechanism
can be seen. 

An example of this situation is shown in Figure 3. The orange pair
is going towards the assigned frontier highlighted by the red dot
in Figure 3.a. As this is reached (Figure 3.b), being the number of
frontiers high, the orange pair is split, assigning to each robot
the closest frontier. These are represented as a green dot for the
robot which previously was the follower, and as a pink one for the
leader (Figure 3.c). At this point, each robot goes towards its own
frontier (Figure 3.d). 

\begin{figure}[t]
\subfloat[The orange pair is moving towards the frontier, depicted as a red
dot]{\includegraphics[width=6cm]{\string"/home/alex/Scrivania/desktop/Tesi/Capitoli/Schermata da 2020-03-29 14-47-53\string".eps}

}\quad{}\subfloat[The pair reaches the assigned frontier]{\includegraphics[width=6cm]{\string"/home/alex/Scrivania/desktop/Tesi/Capitoli/Schermata da 2020-03-29 14-48-09\string".eps}

}

\subfloat[The pair is split assigning the follower to the green dot and the
leader to the pink one]{\includegraphics[width=6cm]{\string"/home/alex/Scrivania/desktop/Tesi/Capitoli/Schermata da 2020-03-29 14-48-20\string".eps}

}\quad{}\subfloat[The two robots move each one towards its own assigned frontier]{\includegraphics[width=6cm]{\string"/home/alex/Scrivania/desktop/Tesi/Capitoli/Schermata da 2020-03-29 14-48-28\string".eps}

}\caption{Buddy split}
\end{figure}

If a robot split from its buddy finds another branching point, a couple
from the idle set is called and assigned to one of the two frontiers,
while the single robot goes towards the other. This clarifies why
the buddy system can be seen as a mix of the reserve mechanism and
the divide and conquer mechanism. It both keeps the idle set waiting
at the initial location until the moment at which it is needed, as
reserve, and each leader goes with a follower from which splits when
a branching point is met, similarly to divide and conquer. In this
way, when a follower is turned into active, it is already pretty near
to its frontier, providing a similar advantage of divide and conquer,
without the disadvantage of having the whole team moving close. 

\subsection{Proactive mechanisms}

In \cite{Cattaneo2017}, three modifications for the mechanisms presented
in \cite{Rogers2013} are introduced. As stated previously, they focus on
enhancing the proactivity of the idle set. The way in which this is
done is pretty straightforward for what concerns reserve and buddy
system, while it is a little more tricky for divide and conquer. The
mechanisms proposed are named \emph{proactive reserve, proactive buddy
system, }and \emph{side follower. }

The main problem with reserve mechanism is that, once a robot from
the idle set is turned into active, it has to move from the initial
location to its assigned frontier. The distance it has to travel may
be lower if the robot is moved to a nearer position while it is still
idle. This is exactly what this modified mechanism tries to do by
moving the idle robots to the barycenter of the polygon whose vertices
are active robots positions. In this way, they are likely to be nearer
to the newly detected frontiers which have to be explored. 

The buddy system faces a similar problem as the reserve mechanism:
the robots in the idle set wait to be turned into active at the initial
location. Thus, the proactive version of the buddy system moves the
idle pairs at the barycenter of the polygon formed by the active agents
locations, for the same reason explained above concerning the proactive
reserve.

The problem of divide and conquer is different, being related to the
interference caused by moving the whole team of robots together. Therefore,
the team of robots is split into groups of three at the beginning
of the exploration and roles are assigned to them. The central robot
is the leader, the right one is the right follower and the left one
is the left follower. These roles are statically determined and never
modified. Moreover, they affect the assignment of frontiers to the
group members, being the frontiers along the direction of the movement
assigned to the leader, the ones on its right assigned to the right
follower and the ones on the left to the left follower symmetrically.
In this way, there is a trade-off between the interference caused
by the number of robots moving together and the distance from the
frontiers, reducing the first one without affecting too much the second
one. 
