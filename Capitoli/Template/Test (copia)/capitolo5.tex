\chapter{Architettura del sistema}
\label{capitolo5}
\thispagestyle{empty}

This thesis is aimed at exploiting measures related to the environment
into a proactive allocation of idle agents. This tries to be a further
improvement of the mechanisms proposed in \cite{Rogers2013}, with
respect to \cite{Cattaneo2017}. The proposed mechanisms are analyzed in
detail in this chapter, giving particular attention to how the various
elements presented before, namely graphs and centrality measures,
are employed. 

At first, a common structure for the coordination mechanisms is defined,
and it is shown how a different implementation of the different functions
allows characterizing each one. This is done starting from reserve
and buddy system, as presented in \cite{Rogers2013}, being the building
blocks of the other mechanisms proposed. Then, the focus shifts to
the proactive versions developed by \cite{Cattaneo2017}.

After this, it is presented how the proposed proactive mechanisms
modify some components of this common structure to include the use
of topological aspects. 

\section{Base mechanisms}

Reserve and buddy system have been introduced intuitively in the previous
chapters, and here a more detailed look at how they are implemented
is given. Recalling that the buddy system is conceived as a mix from
reserve and divide and conquer mechanisms, it is clear why they share
some algorithms. 

The general structure for the coordination mechanisms analyzed can
be roughly summarized of four elements:
\begin{itemize}
\item planning function;
\item activation function;
\item goal function;
\item proactivity function.
\end{itemize}
For some aspects, the planning function works as a common interface
provided by the mechanism. As the simulation requires an agent to
move, the step to take is returned by this function. It calls the
appropriate function coherently with the state of the agent, whether
it is active or idle. It also initializes the exploration by setting
the starting agent and forming the active and the idle sets.

The remaining three functions allow characterizing a coordination
mechanism. They enclose the logic about the assignment of frontiers
to active agents, what triggers idle agents to be turned into active
and where they are kept until that moment.

In particular, the proactivity function is the focus of this work.
In the previous chapter, the importance of graphs and centrality measures
has been introduced. How these are included in the proposed coordination
mechanisms is shown in the last section, giving also attention to
some optimization performed to avoid an excessive impact of their
computation. 

\subsection{Reserve}

Reserve mechanism divides the team of robots into two sub-teams, one
composed of active agents and the other composed of idle ones. This
separation is done by the planning function as soon as the exploration
begins, once the first set of frontiers is computed. 

At first, the distance of each agent from each frontier is retrieved,
then the agent of the pair with the lowest distance is set as the
starting agent and assigned to that frontier. This assignment is iterated
until every frontier is assigned to an agent or vice versa. The set
of assigned agents composes the active set, while the remaining agents
if any, form the idle set. After this initialization step, the planning
function is invoked every time an agent either has reached its goal
or a certain amount of time has passed since the last planning. In
both cases, it is in charge of calling the right function among the
activation function, the goal function or the proactivity one.

Once an agent is active, which frontier is assigned to it is computed
by the goal function. To do this, the planning function provides it
with the updated list of frontiers. The assignment is simply done
considering the closest frontier to the agent location. 

The activation function can be invoked only for agents of the idle
set, being the function aimed at turning them into active. This function
checks whether there are any not assigned frontiers and if so, the
agent is turned into active and assigned to the closest one. If this
is not the case and all the frontiers are assigned, the proactivity
function is called. 

The reserve policy is that the idle set waits at the initial location
and thus, the proactivity function always returns the position of
the agent. In this way, agents of the idle set stay still, until the
activation function assigns them to a frontier. 

\subsection{Buddy system}

Buddy system proceeds in a way similar to the reserve one. This method
is characterized by the use of paired robots and each robot in the
pair is the buddy of the other, from this the name of the mechanism.
Pairs are created before the exploration begins, by assigning a role
to each robot of the team. The roles are two, either an agent is a
leader or a follower and they are assigned to split the team into
two halves. In the case of odd teams, the extra agent is assigned
leader role.

Apart from this division into leaders and followers, the team is also
split into an active set and an idle set. The active set is composed
of leaders assigned to frontiers and their buddies. The idle set is
composed of the pairs waiting at their initial locations.

The first call to the planning function selects the starting pair.
This is done similarly to the reserve, by looking for the robot closest
to a frontier. If it is a leader, it is assigned to that frontier.
If it is a follower, its buddy is retrieved and being it the leader,
it is assigned to that frontier. As for reserve, this approach is
iterated as long as there are no more leaders or frontiers to assign.
All the leaders assigned to a frontier and their buddies compose the
active set. The idle set is composed of the pairs whose leader is
not assigned to a frontier if any.

Further calls to the planning function distinguish whether the calling
agent is a leader or a follower. In the former case, it simply invokes
the goal function, while in the latter it checks whether the pair
needs to be split. This happens when the two closest frontiers are
enough distant one from the other. This is a branching point and the
pair is split. The leader is assigned to the closest. The follower
is now considered a leader itself and is assigned to the second closest
frontier. If that is evaluated not to be a branching point, then the
goal function for the follower is called. In fact, the goal function
is implemented differently whether the considered agent is a leader
or a follower. 

The goal function for leaders is the same as reserve. It assigns the
agent to the closest frontier, while the goal function for the follower
moves the agent towards the frontier assigned to its leader. This
makes the pair go together in the direction of the assigned frontier.

The activation function for the buddy system follows the same principle
of the reserve one. If there are any not assigned frontiers, the closest
idle leader to each one is computed and turned into active by assigning
that frontier to it. This has the side effect of turning into active
also its buddy because the goal function will now make the follower
follow its leader, rather than waiting at the initial position. 

Buddy system keeps the idle set waiting at the initial position. Therefore,
when called by an agent, the proactivity function simply returns the
current location of the agent. 

\subsection{Benchmark proactive mechanisms}

The proactivity function in the previous cases consists only of returning
the position of the agent calling it. This makes the agents stay still
at their initial locations until the activation function turns them
into active. As discussed in the previous chapters, this makes the
average distance between the agents and the assigned frontier higher,
penalizing performance. To overcome this, a new proactivity function
has been proposed in \cite{Cattaneo2017}. Robots of the idle set are moved
to a position likely to be nearer the future assigned frontier and
this speeds up exploration. Experimental results show this modification
to actually be an enhancement in the case of reserve. The proactive
version outperforms the base version on almost every environment on
which it has been tested. The proactive buddy system, on the contrary,
is affected less by the proactivity. 

The previously defined functions are the same for both the algorithms.
The only difference is in the proactivity function. For both proactive
reserve and proactive buddy system, the proactivity function assigns
to the agent a goal location computed as the barycenter of the locations
of the active agents. It can be formalized as 
\[
G_{t}(a)=\frac{1}{\mid A\mid}\sum_{a'\in A}P_{t}(a')
\]

recalling the notion from the previous chapter, where $a$ and $a'$
are agents, in particular, $a$ is an idle agent and $a'$ iterates
over $A$, the set of active agents. $P_{t}\left(a'\right)$ is a
function providing the location of the agent $a'$ at the time $t$
and $G_{t}\left(a\right)$ is the goal location assigned to the idle
agent. The subscript $t$ for the goal function is possible since
the location of the active agents at time $t$ are used to compute
the barycenter. There is no delay in the propagation of the information. 

This proactivity function is characterized only by the actual location
of active agents. The environment is not explicitly included in this
formula. However, it affects indirectly where the robots are moved 
as robots are moving into it. To take its features into account directly, 
an approach based on graphs and centrality measures is proposed in the following section, 
characterized by a different implementation of the proactivity function. 

\section{Proposed proactive mechanisms}

The proposed proactive mechanisms differ from the previously defined
ones mostly for the proactivity function. There are no other conceptual
differences in the other elements of the structure. Even if a small
variation to the implementation of the goal function is introduced,
as will be pointed out in the next section.

The proactivity function is modified in a way to include the computation
of either the closeness or the betweenness and to use their values
to obtain the proactivity point. Independently from the measure needed,
the proactivity function follows the same steps. At first, it is checked
if the graph has been modified since the last computation. If not,
the proactivity point returned is the same as the previous step, avoiding
useless computations. Otherwise, a new point has to be evaluated.
This is done by determining the value of the considered metric for
the nodes of the graph and then retrieving the most central one, i.e.,
the one with the highest value of the metric. If more than one node
is found, the returned point is their barycenter. However, this case
has a very low frequency and the number of nodes never exceeded two,
during the experiments. 

As presented in the previous chapter, the proposed proactive mechanisms
exploit two different types of graphs, combined with two centrality
measures. All their combinations have been tested both applied to
the reserve and the buddy system, for a total of four different versions
for each one. 

In the following sections, it is presented how the mechanisms have
been adapted to include the generation of the graphs and some optimizations
aimed at allowing an efficient computation of the centrality measures.
This was needed because of the high complexity of the algorithms used,
which made the average simulation cycle to greatly increase its duration,
particularly in the more advanced stages of the exploration. 

\subsection{Graph building process}

In the proposed mechanisms, there are two types of graphs tested.
As already stated before, they are referred to as topological graph
and visibility graph. Both have been implemented and tested to check
how differently their use would affect performance. Their structures
are built to enforce different aspects of the environment. The topological
graph is built on the notion of navigability, that is it aims at capturing
the connections among spaces. The visibility graph is based on the
concept of visibility, which is more related to the possibility of
sensing a certain location from another one.

Both graphs have been implemented employing adjacency lists \cite{Cormen2010}. 
Recalling that the map is modeled as an occupancy grid, each node is characterized
by its coordinates in the global coordinate system. It also holds a list 
containing all the nodes linked to it and the distance separating them. 
This has been preferred over a representation through adjacency matrices because
both the graphs tend to be sparse, in particular the topological one.
To confirm this, the average degree of the nodes of the topological
graph is two, whereas it has a high variability for the visibility
one. However, even in open environments, where the number of nodes
visible from a node is higher, the average degree is always lower
than half of the total number of nodes. 

In the following, how these graphs are built is shown in detail, starting
from the topological graph, then focusing on the visibility one. 

\subsubsection{Topological graph}

The topological graph is a graph isomorphic to the one used by the
navigation system to compute the paths followed by robots \cite{Spirin2015}. 
In the following, this last one is referred to as navigation graph to avoid
confusion. Whether it is a frontier for an active agent or a proactivity
point for an idle one, every time the path from an agent position
to its goal is needed, the navigation graph is checked. In case the
time since the last update is higher than a threshold or the occupancy
grid of the environment has changed, it is recomputed. This is done
in the following way. The obstacle cells of the occupancy grid are
progressively enlarged until the skeleton of the free space is found.
It is then discretized into a set of nodes that are going to be the
nodes of the navigation graph. The first kind of nodes correspond
to branching points, then it looks for points filling the gaps among
them. These points are added as nodes if the distance from the closest
node is higher than a certain fixed value. At this point, the set
of nodes for the navigation graph is complete but a further pruning
is applied to remove the ones too close to an obstacle. Each point
of the occupancy grid is then mapped to the nearest node and this
provides a partition of the map into different polygons, each one
associated with a node. Two nodes are adjacent if the polygons associated
with them share one side and this allows finding the edges of the
graph, completing its construction. 

Once the navigation graph is computed, the topological graph is built
accordingly of the same set of nodes with the same adjacency relations.
The only difference is that the edges of the navigation graph do not
keep track of the distance between nodes, while the edges of the topological
graph do. Indeed, the topological graph is a weighted undirected graph,
where the weight function is a distance function. Given the structure
of this graph, the difference between Euclidean distance and the effective
length of the path between the two nodes is negligible, thus the first
one has been applied for efficiency reasons. 

The update of this graph is equivalent to the computation of a new
navigation graph and is done every time a path for an agent is needed.
Thus, referring to the general structure of a coordination mechanism
presented before, it may be done by each of the four functions because
each one implies the computation of a path for the agent. This ensures
the topological graph to be updated frequently. 

\subsubsection{Visibility graph}

As presented in Chapter 4, the visibility graph is composed of pose
nodes and frontier nodes. Pose nodes map the positions assumed by
the robots at various time instants. Frontier nodes are used to include
the location of frontiers known at a certain moment into the graph.

It is initialized at the beginning of the exploration, considering
as pose nodes the set of initial positions of the robots. The frontier
nodes are included as soon as the first scan provides the first set
of frontiers. After this, it is never rebuilt from scratch, like happens
for the topological graph, rather it is progressively updated to match
the movements of the robots and the disclosure of new frontiers. This
holds because its construction is carried on entirely during the calls
to the goal function. In both the reserve and the buddy system, this
function comprises the update of the list of frontiers to properly
assign robots. This provides the possibility to remove old frontier
nodes and include the new ones. During this step, also the actual
locations of active robots are used to generate nodes of the graph,
the pose nodes, and to add them. Once all these nodes have been included,
edges are generated. They model the notion of visibility, thus an
edge links two nodes if and only if a robot located in one is able
to perceive the other one. This condition is way more restrictive
than the navigability one, on which the topological graph is based.
A simple example of this is the case of two nodes located on opposite
sides of a wardrobe. It is clearly possible to define a path connecting
them, but the sensors of a robot placed on one side can not provide
measurements about what is on the other side. Moreover, edges built
in this way are straight lines and as for the topological graph, the
distance function providing the weight to them is the Euclidean distance. 

\subsection{Centrality measures}

The main problem with closeness and betweenness is related to their
computational complexity. They require the knowledge of the shortest
paths connecting each pair of nodes and their length. This is particularly
true for betweenness, while closeness only needs the last one. As
the size of the graph grows, it may be difficult to compute the proactivity
point in a reasonable amount of time. For this reason, two major optimizations
have been performed. The first one is to compute the matrix of the
distances between each pair of nodes in an efficient way and store
it, allowing to avoid the repeated computations of the length of the
shortest paths. The second one relates to how betweenness is computed. 

\subsubsection{Distance matrix}

The distance matrix is a square matrix containing the distances between
each pair of nodes of a graph. Both the topological and the visibility
graphs are weighted undirected graphs, thus it is useless to have
a square matrix, being it symmetric by construction. Moreover, the
weights considered are always non-negative, for this reason, the distance
of a node from itself is always zero. This allows reducing the size
of the matrix from $N\times N$ to an $\left(N-1\right)\times\left(N-1\right)$
triangular matrix, where $N$ is the number of nodes. 

This data structure is essential in speeding up the computation of
closeness. Recalling that the closeness of a node is defined as the
inverse of the average distance of it from all the other nodes, it
can be computed as
\[
C\left(x\right)=\frac{1}{avg\left[row\left(x\right)\cup col\left(x\right)\right]}
\]

where $row\left(x\right)$ and $col\left(x\right)$ provide respectively
the elements of the row and the column of the distance matrix associated
to node $x$. While $avg$ is simply the function computing the average.
The union is to ensure considering each element, being the distance
matrix triangular. Thus, in a single sweep of the whole matrix, the
value of the closeness of each node can be computed. 

The main issue about the distance matrix is in its construction. It
is an instance of the All Pairs Shortest Path problem, that is the
problem of finding the shortest paths linking each pair of nodes.
An algorithm able to solve this is the Floyd-Warshall algorithm \cite{Cormen2010}. 
It is based on the concept of \emph{intermediate nodes}, which is the
set of nodes composing a path without the starting and the arrival
nodes. Consider a graph $G$ which nodes are $V=\left\{ 1,2,\ldots,n\right\} $
and a subset $\left\{ 1,2,\ldots,k\right\} $ of $V$ for a generic
$k$. Let also $i$ and $j$ be two nodes of $G$ and consider all
the paths connecting them, composed only of vertices in $\left\{ 1,2,\ldots,k\right\} $.
Let $p$ be a shortest path connecting $i$ and $j$. If $k$ is an
intermediate node of $p$, then the length of the path from $i$ to
$k$, plus the one from $k$ to $j$ is lower than the one from $i$
to $j$ having nodes in $\left\{ 1,2,\ldots,k-1\right\} $ as intermediate
nodes. On the contrary, if $k$ is not an intermediate node, then
this last path is shorter. By progressively increasing the value of
$k$ to consider the whole set of nodes as possible intermediate nodes,
it is possible to find all the shortest paths for each pair of nodes
of the graph.

The complexity of the algorithm is $\Theta\left(N^{3}\right)$. This
is manageable in the case of the topological graph because $N$ is
limited by the graph building algorithm. On the other hand, the number
of nodes in the visibility graph is much higher because no pruning
is applied. The only device consists of forcing a minimum distance
within pose nodes. However, as the exploration goes on, $N$ might
be sufficiently high to make the application of the Floyd-Warshall
algorithm heavy. To overcome this issue, two factors allowed to come
up with an efficient solution. 

First of all the visibility graph is updated at every step and is
never built from scratch, after the initialization. The second main
aspect is that the Floyd-Warshall algorithm is an example of dynamic
programming, thus the solution provided for the graph at time $t$
can be used as a starting point to compute the solution for the graph
at time $t'>t$. In fact, paths computed with a reduced set of intermediate
nodes, suppose it to be $\left\{ 1,2,\ldots,k-1\right\} $, are the
shortest paths possible if the graph had node in $\left\{ 1,2,\ldots,k-1\right\} $.
Then, the extension of the intermediate nodes to $\left\{ 1,2,\ldots,k\right\} $
provides the optimal solution for the graph with nodes in $\left\{ 1,2,\ldots,k\right\} $.
This idea can be applied by looking at the graph at time $t$ like
the one with intermediate nodes in $\left\{ 1,2,\ldots,k-1\right\} $,
while the one at time $t'>t$ as the one with intermediate nodes in
$\left\{ 1,2,\ldots,k\right\} $. 

In this way, the distance matrix of the visibility graph is computed
at the beginning of the exploration and consequently updated as new
nodes are added to the graph. The complexity of each update is linear
in the number of nodes, including the new ones. 

Apart from computing all the distances between each pair of nodes,
it is also possible to reconstruct the shortest paths linking them.
Which modifications need to be done and how this is exploited are
discussed in the following section.

\subsubsection{Betweenness}

Efficient computation of betweenness requires the preemptive computation
of the shortest paths between each pair of agents. Assume to have
these stored in a matrix, that, similarly to the distance matrix,
is an $\left(N-1\right)\times\left(N-1\right)$ triangular matrix.
Even if this is available, finding the betweenness $B\left(v\right)$
of a node $v$ requires to go through each pair of nodes $s$ and
$t$, such that $s\neq v\neq t$, and retrieve the list of shortest
paths linking them. This list needs to be iterated to count the number
of shortest paths going through $v$ and their total number for that
pair of nodes. At this point, the ratio of these two quantities can
be computed and the result has to be summed with the value for each
other selection of $s$ and $t$. Similarly to the application of
the Floyd-Warshall algorithm, this algorithm may be a problem for
the case of a visibility graph, where the number of nodes $N$ might
increase enough to make the computation unfeasible in a reasonable
amount of time. 

The main aspect of improvement is in the computation of the number
of shortest paths from $s$ to $t$ going through $v$, which in the
previous chapter was referred to as $\sigma_{st}\left(v\right)$.
This is made possible by the \emph{Bellman criterion} \cite{Brandes2001}, 
which states that a node $v$ of a graph lies on a shortest path between two nodes
$s$ and $t$, if and only if $d\left(s,t\right)=d\left(s,v\right)+d(v,t)$.
Even if apparently obvious, this criterion allows to compute $\sigma_{st}\left(v\right)$
as 
\[
\begin{cases}
0 & if\ d\left(s,t\right)<d\left(s,v\right)+d\left(v,t\right)\\
\sigma_{sv}\cdot\sigma_{vt} & otherwise
\end{cases}
\]

where $\sigma_{sv}$ is the value of the count of the shortest paths
from node $s$ to node $v$, and $\sigma_{vt}$ is the analogous from
$v$ to $t$. The idea is that if $v$ is along a shortest path from
$s$ to $t$, then $\sigma_{st}\left(v\right)$ can be computed as
the number of shortest paths from node $s$ to node $v$ multiplied
by the number of the ones from $v$ to $t$, to include all the possible
combinations.

On this, the betweenness can simply be computed in the same way as
above. This also allows avoiding the storage of the whole list of
shortest paths for each pair of nodes because it is only needed their
count and the distance matrix. The Floyd-Warshall algorithm can be
suitably modified to fill the matrix of counts by updating it as new
shortest paths are found. This can be done while checking if the length
of the path going through the considered intermediate node is lower
than the previously known path. If the new path is lower, then the
counter is reset to one. If the length is equal, then the counter
is increased by one. If the new path is longer, then the counter remains
equal because the new path is not a shortest path. 

Once both these matrices are completed, computation of the betweenness
for the whole set of nodes requires for each node to iterate over
all the possible pairs of other nodes. Thus, the computation has cubic
complexity in the number of nodes but it is independent of the length
of the paths considered. On the contrary, this would affect the computation
in the case the entire paths had to be scanned to check whether node
$v$ was present or not.

\subsubsection{Reduced node-set}

The optimizations provided before to speed up the computation of the
measures, in particular for betweenness, work well in practice. This
holds both for the topological graph and for the visibility graph,
where the number of nodes is way higher, being even two to three times
it. In some cases, this ratio can also go up to four times, producing
an obliged impact on the computation. After all, the complexity holds
a cubic relation with the number of nodes. 

Moreover, another aspect taken into consideration is related exclusively
to the visibility graph. The value of both the centrality measures
for frontier nodes is almost uninformative. In fact, the value of
betweenness is forced to be zero because no shortest paths between
a pair of nodes go through a frontier. On the other hand, the value
of closeness is always small as compared to the other value of closeness.
This can be explained easily considering that frontier nodes are marginal
nodes of the graph, because of their definition. Furthermore, being
the sensing range way higher than the speed of the robot, the distance
between pose nodes is lower than the distance between them and frontier
nodes. Thus, the impact of frontier nodes on the value of closeness
is negligible. 

To deal with both these theoretical and empirical considerations,
it has been decided to restrict the set of nodes for which centrality
measures are computed. The reduced node-set $R$ is then composed
of pose nodes $n$ such that $n$ either is adjacent to a frontier
node $f$ or is adjacent to a pose node $n'$ adjacent to $f$. To
have a clear definition of how this set is composed, it is worth defining
a function $adj(n)$ for a node $n$ of a graph returning the set
of nodes adjacent to $n$ and indicating as $F$ the set of frontier
nodes. On this, an auxiliary set $A$ can be defined as the set of
nodes adjacent to a frontier node, namely
\[
A=\bigcup_{f\in F}adj\left(f\right)
\]

Then, the definition of $R$ can be rewritten as 
\[
R=\left(A\:\cup\:\bigcup_{a\in A}adj\left(a\right)\right)\setminus F
\]

$R$ composed in this way allows a trade-off between considering the
whole set of nodes and just the ones adjacent to a frontier node.

Thanks to all the tricks exposed so far, it has been possible to reduce
the average simulation cycle time of the proposed mechanisms to values
comparable to the ones of the mechanisms exploiting the barycenter. 
